\documentclass[12pt]{article}

\begin{document}
\huge 
\textbf{CTA200H Final Project}
\Large

Student: Mathew Bub

Supervisor: Jo Bovy

May 18, 2018

\normalsize

\section*{Question 1}
The first question asks us to write a function that takes the name of a star and returns a galpy.Orbit object. To do so, we must first retrieve data about the star's position and velocity from the \textit{Gaia} DR2 catalogue. However, \textit{Gaia} does not store the names of stars. Thus, each star name must first be correlated with a position (in right ascension and declination) in order to query the \textit{Gaia} catalogue.

My function first queries the SIMBAD catalogue for the star's name, retrieving a table containing the right ascension, declination, parallax, proper motion, and radial velocity (if available) for that star. This data is used in the \verb EPOCH_PROP_POS  function built into the \textit{Gaia} Archive ADQL query tool. The function returns the right ascension and declination of the star propagated from a reference epoch to a later time. The \textit{Gaia} catalogue is then queried in a 1 arcsecond radius around these propagated coordinates, which returns the data for the desired star in a table.

\end{document}